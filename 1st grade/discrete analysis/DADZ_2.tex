\documentclass[12pt]{article}

\usepackage{baseset}
\DeclareSymbolFont{operators}{OT1}{ntxtlf}{m}{n}
\SetSymbolFont{operators}{bold}{OT1}{ntxtlf}{b}{n}
\newcommand{\RomanNumeralCaps}[1]
{\MakeUppercase{\romannumeral #1}}
\usepackage{enumitem}

\author{Красоткина Виктория}

\title{Дискретный анализ. Домашнее задание 2}

\date{2022 г.}

\begin{document}
	
	\maketitle
	\tableofcontents
	\thispagestyle{empty}
	\newpage
	
	\section{Комбинаторика \RomanNumeralCaps{3}}
	\begin{enumerate}[label={\textbf{\arabic{section}.\arabic*}}]
		\item Сколькими способами можно закрасить клетки таблицы $3\times 4$ так, чтобы незакрашенные клетки содержали или верхний ряд, или нижний ряд, или две средних вертикали?
		
		\textbf{Решение}
		
		Всего $12$ клеток. Для первых двух случаев существует по $2^8$ вариантов (так как есть $2$ состояния клетки -- закрашена и не закрашена -- и $8$ вакантных клеток) раскраски таблицы. В случае незакрашенных вертикалей есть $6$ свободных клеток, значит $2^6$ вариантов.
		
		Рассмотрим случаи пересечения: $2^4$ вариантов, когда выполняются первые два условия, по $2^4$ вариантов, когда выполняется первое + третье и второе + третье условия. $2^2$ вариантов, когда выполняются все три условия.
		
		Тогда по формуле включения-исключения:
		$$
		N = (2^8 + 2^8 + 2^6) - (2^4 + 2^4 + 2^4) + 2^2 = 532
		$$
		Ответ: \textbf{532 способа}
		
		\item Для полета на Марс набирают группу людей, в которой каждый должен владеть хотя бы одной из профессий повара, медика, пилота или астронома. При этом в техническом зада-
		нии указано, что каждой профессией из списка должно владеть ровно $6$ человек в группе. Кроме того указано, что в группе должен найтись ровно один человек, владеющий всеми этими профессиями; каждой парой профессий должны владеть ровно $4$ человека; каждой тройкой -- ровно $2$. 
		
		Выполнимо ли такое техническое задание?
		
		\textbf{Решение}
		
		Обозначим за $A_1$ множество поваров, за $A_2$ -- медиков, $A_3$ -- пилотов, $A_4$ -- астрономов. Запишем условия:
		$$
		\begin{cases}
			|A_1| = |A_2| = |A_3| = |A_4| = 6 \\
			|A_1\cap A_2\cap A_3\cap A_4| = 1 \\
			|A_1\cap A_2| = |A_1\cap A_3| = |A_1\cap A_4| = |A_2\cap A_3| = |A_2\cap A_4| = |A_3\cap A_4| = 4 \\
			|A_1\cap A_2\cap A_3| = |A_1\cap A_2\cap A_4| = |A_1\cap A_3\cap A_4| = |A_2\cap A_3\cap A_4| = 2
		\end{cases}
		$$
		Найдем количество людей в группе, чтобы условия выполнялись:
		$$
		N(|A_1\cup A_2\cup A_3\cup A_4|) = (6 + 6 + 6 + 6) - (4 + 4 + 4 + 4 + 4 + 4) + (2 + 2 + 2 + 2) = 7
		$$ 
		Допустим, первые $6$ из $7$ человек -- повара. Тогда среди медиков есть как минимум $5$ медиков, а это противоречит условию $|A_1\cap A_2| = 4$.
		
		Значит, \textbf{условие невыполнимо.}
		
		\item Пусть $A$ и $B$ -- конечные непустые множества, и $|A| = n$. Известно, что число инъекций из $A$ в $B$ совпадает с числом сюръекций из $A$ в $B$. Чему равно это число?
		
		\textbf{Решение}
		
		Инъекция $\Rightarrow$ $|B|\geq|A|$
		
		Сюръекция $\Rightarrow$ $|B|\leq|A|$
		
		По условию число инъекций из $A$ в $B$ совпадает с числом сюръекций из $A$ в $B$, значит они обе существуют, и $|A| = |B| = n$. Таким образом, нужно понять, сколько существует способов сопоставить элементы $B$ элементам $A$. Очевидно, $n!$. Это и есть ответ.
		
		\item Пусть $X = \{1,\dots, n\}$. Найдите число способов взять $k$ подмножеств $X_1,\dots,X_k$ множества $X$ таких, что $X_1\subseteq X_2\subseteq\dots\subseteq X_k$.
		
		\textbf{Решение}
		
		Если какой-то элемент из множества $X$ принадлежит множеству $X_i$, то он принадлежит и $X_j$, где $j > i$. Значит, каждому элементу из $X$ можно поставить в соответствие число $i$ -- номер множества, где он всречается первый раз:
		$$
		i\in\{1,2,\dots,k+1\}
		$$
		Найдем число способов поставить $i$ в соответствие элементу из $X$: оно будет равно $(k+1)^n$ -- это и есть ответ на вопрос задачи.  
		
		\item В классе $20$ учеников, каждый из которых дружит ровно с шестью одноклассниками. Найдите число таких различных компаний из трёх учеников, что в них либо все школьники дружат друг с другом, либо каждый не дружит ни с одним из двух оставшихся.
		
		\textbf{Решение}
		
		Количество компаний из трех человек.
		$$
		N_1 = C_{20}^3 = \frac{20!}{3!17!} = \frac{18\cdot19\cdot20}{6} = 1140
		$$
		Найдем число компаний, в которых хотя бы один человек с кем-то дружит, но не каждый с каждым:
		$$
		N_2 = \frac{20\cdot6\cdot(19-6)}{2} = 780
		$$
		Тогда искомое количество:
		$$
		N = N_1 - N_2 = 1140 - 780 = 360
		$$
		
		\item Найдите количество неубывающих отображений
		$$
		f:\{1,2,\dots,n\}\rightarrow\{1,2,\dots,m\}
		$$
		
		\textbf{Решение}
		
		Пусть $\{x_n\}$ -- неубывающая последовательность такая, что 
		$$
		x_k = f(k)~\forall k\in\{1,2,\dots,n\}
		$$
		$f$ -- неубывающее отражение, значит, $\{x_n\}$ -- неубыающая последовательность. Тогда нам остается найти количество способов выбрать $n$ из $m$ элементов с повторениями, то есть
		$$
		N = C_{n+m-1}^{m}
		$$
		
		\item Чего больше, разбиений $n$-элементного множества на не более чем $k$ подмножеств или разбиений $(n+k)$-элементного множества на ровно $k$ подмножеств?
		
		\textbf{Решение}
		
		Предположим, что мужчины и женщины различимы, места за столом тоже различимы. Если женщины займут чётные места $n!$ способами, то мужчины будут занимать нечетные места тоже $n!$ способами и наоборот. Тогда
		$$
		N = 2\cdot(n!)^2
		$$
		
		\item Функция неубывающая, если $x \leq y$ влечет $f(x) \leq f(y)$. Найдите количество 
		\begin{enumerate}[label=\textbf{\alph*)}]
			\item неубывающих инъекций $f$ : $\{1,\dots,n\}\rightarrow\{1,2,\dots,m\}$
			\item неубывающих сюръекций $f$ : $\{1,\dots,n\}\rightarrow\{1,2,\dots,m\}$
		\end{enumerate}
	
		\textbf{Решение}
		\begin{enumerate}[label=\textbf{\alph*)}]
			\item Мы выбираем $n$ элементов из $m$ $C^n_m$ способами, но тут уже не делаем перебор всех возможных перестановок, т.к. нам удовлетворяет ровно одна, т.к. все числа во множестве $B = \{1,2,\dots,m\}$ различны.
			\item Все элементы множества $A$ -- шары, а элементы множества различные ящики в кол-ве m штук. Тогда воспользуемся формулой шаров и перегородок: $C^{m-1}_{n-1}$.
		\end{enumerate}
		
		\item Найдите сумму:
		$$
		n^n - C^1_n(n-1)^n+C^2_n(n-2)^n+\dots+(-1)^nC^n_nN_n
		$$
		
		\textbf{Решение}
		
		Заметим, что данная сумма эквивалентна формуле включения-исключения. Пусть $\{a_1,a_2,\dots,a_n\}$ -- свойства. $a_i$ -- элементу $y_i$ не сопоставлен $x$. Значит, $N(\overline{a_1},\overline{a_2},\dots,\overline{a_n})$ -- число таких отображений, что каждому $y_i$ сопоставлен $x$.А раз у каждого $x$ свой $y$, то существует $n!$ способов их распределить. Значит, $N(\overline{a_1},\overline{a_2},\dots,\overline{a_n}) = n!$ -- и это же ответ на вопрос задачи. 
	\end{enumerate}
	\newpage
	\section{Неориентированные графы}
	\section{Деревья и раскраски}
	\section{Ориентированные графы}
	\section{Бинарные отношения}
	\section{Производящие функции}
	
\end{document}